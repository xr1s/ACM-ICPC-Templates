\section{战术}
\subsection{上机检查}
\begin{itemize}
  \item 机器一秒速度, 能做多少次相加取模?
  \item 栈大小, 可行递归深度, 扩栈代码可用否?
  \item \lstinline{__int128}, \lstinline{__float128} 可用否?
  \item \lstinline{#pragma} 架构优化可行否?
  \item GNU 扩展库 (\lstinline{rope}, \lstinline{pb_ds}) 可用否?
\end{itemize}

\subsection{战前准备}
\begin{itemize}
  \item 确定厕所位置, 不要在赛场找厕所浪费时间.
  \item 薄荷糖, 提神醒脑.
  \item 零食, 充饥, 切忌太多, 不然易困.
  \item 水, 解渴, 切忌太多, 不然尿急.
\end{itemize}

\subsection{战时策略}
\begin{itemize}
  \item 题意一定要经过两个人确认.
  \item 解题思路一定要经过一个队友确认.
  \item 交给队友的信息一定要对准确度负责.
  \item 注意数据大小, 时限, 内存, 是否应当启用读入优化.
  \item 注意 clarification, 读完题目看一遍, 写完题目看一遍.
  \item 复杂的逻辑先在纸上写好, 以防上机越写越乱.
  \item 不要放空自己, 不要慌, 就算到了最后十分钟也要全神贯注.
\end{itemize}
