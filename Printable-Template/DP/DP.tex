\section{动态规划}
\subsection{最长公共子序列}
对于长度分别为 $n,m$ 的串 $s,t$, 求它们长度最长的公共子序列.

时空复杂度均为 $O(nm)$. 使用滚动数组可以降低一维, 空间复杂度 $O\left(\min\{n,m\}\right)$.

\lstinputlisting{DP/LCS.hh}

如需求出最长公共子序列本身, 在求出的 \lstinline{lcs} 数组中找到那些被加一的位置即可.

当其中一个串中的字符不重时, 可以将字符按位置递增编号, 则可以在另一个串中球最长不下降子序列, 时间复杂度降为线性对数级.

\subsection{最长上升子序列}
对于长为 $n$ 的串 $s$, 求最长上升子序列.

\lstinputlisting{DP/LIS.hh}

\begin{tabular}{@{}>{\textbullet}cll@{}}
  & 最长上升子序列:   & \lstinline|LIS(s, t, less<>{})|.          \\
  & 最长不下降子序列: & \lstinline|LIS(s, t, less_equal<>{})|.    \\
  & 最长下降子序列:   & \lstinline|LIS(s, t, greater<>{})|.       \\
  & 最长不下降子序列: & \lstinline|LIS(s, t, greater_equal<>{})|. \\
\end{tabular}

\subsection{最大子串和}
对于一个长为 $n$ 的串, 求出最大子串和. Kadane 算法, $O(n)$.

\lstinputlisting{DP/maxSubarray.hh}

\subsection{最大子矩阵和}

给出 $n\times m$ 矩阵, 权值可正可负, 求最大子矩阵和.

枚举上下边界, 上下边界之间的按列求最大子串和, $O\left(nm^2\right)$.

\lstinputlisting{DP/maxSubmatrix.hh}

\subsection{最大全1子矩阵/悬线法}

给定 $n\times m$ 的 01 矩阵, 求最大全 1 子矩阵. 悬线法时间复杂度 $O(nm)$.

\lstinputlisting{DP/largestSubmatrix.hh}
