\clearpage
\section{数论}

\subsection{最大公约数/最小公倍数}
Euclidean 算法: $O(\log\min\{n,m\})$.

Stein 算法: $O(\log\max\{n,m\})$, 乘除取模比加减位移效率低时 Euclidean 算法高效.

\[\operatorname{lcm}(n,m)=\frac{nm}{\gcd(n,m)}\]

\lstinputlisting{NT/gcd-lcm.hh}

\subsection{扩展 Euclidean 算法}

求方程 $nx+my=\gcd(n,m)$ 的解. 当已经求得一个解 $(x_0,y_0)$, 可得解集 \[\left\{\left(x_0+\frac{km}{\gcd(n,m)},y-\frac{kn}{\gcd(n,m)}\right):k\in\mathbb Z\right\}\]

下面代码中函数返回值 \lstinline{get<0>} 为 $\gcd(n,m)$, \lstinline{get<1>} 和 \lstinline{get<2>} 组成上述方程的一个解.

\lstinputlisting{NT/exgcd.hh}

\subsection{模逆元}

当 $n\perp m$, 对于方程 $nx\equiv1\pmod m$ 求解 $x$.

问题可以转换为 $nx-mk=1$, 则可用扩展 Euclidean 算法求解.

亦可利用 Fermat-Euler 定理 $n^{\varphi(m)}\equiv1\pmod m$, 得到 $n\cdot n^{\varphi(m)-1}\equiv1\pmod m$, 则 $x\equiv n^{\varphi(m)-1}\pmod m$.

扩展 Euclidean 算法不会有整型溢出. 幂次可能会在乘法过程中溢出.

\lstinputlisting{NT/modinv.hh}

\subsection{原根}
求素数 $p$ 的原根 $g$, $p$ 的原根 $g$ 是幂函数 $a^x\bmod p$ 的最小循环节为 $\varphi(p)$ 的数.

$p$ 的原根存在的充要条件为 $p=2,4,p^k,2p^k$, 其中 $p$ 为奇素数, $e$ 为正整数.

无多项式解, 优化解法是枚举原根, 如果某个数对 $p-1$ 所有因数的幂均不为 1, 则该数为原根.

\lstinputlisting{NT/primitive.hh}

\subsection{离散对数}
求满足 $g^x\equiv y\pmod p$ 的最小的 $x$, 其中 $g\perp p$. Baby-step giant-step 算法, 时空 $O(\sqrt p)$.

\lstinputlisting{NT/bsgs.hh}

若有 $g\not\perp p$, 则需要先将其化为互质:
\lstinputlisting{NT/exbsgs.hh}

\subsection{二次剩余}
求解 $x^2\equiv n\pmod p$, Tonelli Shanks 算法, 时间复杂度 $O(\log^2p)$.
\lstinputlisting{NT/tonelli-shanks.hh}

Cipolla 算法, 时间复杂度 $O(\log p)$. (随机地, 概率为 1/2) 找出一个数 $a$ 使得 $(a^2-n)\bmod p$ 不为二次剩余 (即 Legendre 符号为 $-1$). 设 $\omega^2\equiv a^2-n\pmod p$ 为单位复数建立``复数域''.

根据定义有 $(a+b\omega)(c+d\omega)\equiv (ac+bd\omega^2) + (ad+bc)\omega$, 可以进行乘法, 那么可以直接求得 $x\equiv(a+\omega)^{\mathlarger{\frac{p+1}2}}$.

\subsection{高次剩余}
对于 $x^a\equiv n\pmod p$. 解法是求出 $p$ 的原根 $g$, 有 $x\equiv g^y\pmod p$, 因此有 $g^{ay}\equiv n\pmod p$, 问题化为求解离散对数问题.

\subsection{中国剩余定理}

对于 $n$ 个同余方程组 $x\equiv a_k\pmod{m_k}$, 满足 $m_k$ 之间两两互质, 求解 $x$.

设 $\displaystyle M=\prod_{k=1}^nm_k$. 对于每个 $k\in\{1,\dots, n\}$, 设 $\displaystyle M_k=\frac M{m_k}$, $t_k\equiv M_k^{-1}\pmod{m_k}$, 有 \[\displaystyle x\equiv\sum_{k=1}^na_kt_kM_k\pmod M\]

\lstinputlisting{NT/crt.hh}

当 $m_k$ 之间不两两互质时, 需要扩展中国剩余定理.

\lstinputlisting{NT/excrt.hh}

\subsection{素性测试}
Miller Rabin 算法, 下述代码在 \lstinline{uint64_t} 内正确, 复杂度 $O(13\log n)$. 注意可能需要 \lstinline{uint128_t}.
\lstinputlisting{NT/is-prime.hh}

\subsection{Euler 线性筛}
$O(n)$ 找出 $1,\dots,n-1$ 数中的素数, 求解积性函数以及一部分特殊数论函数对于取值为 $1,\dots,n-1$ 的函数值.

下面代码中 \lstinline{mu} 为 Möbius 函数 $\mu$, \lstinline{phi} 为 Euler 函数 $\varphi$, \lstinline{sigma} 和 \lstinline{k} 为除数函数 $\displaystyle\sigma_k(n)=\sum_{d\mid n}d^k$.

注意每个合数都只会被其最小的素因子筛去. 因此求除数函数 $\sigma$ 时为了处理当前筛到的合数为某个素数倍数的情况, 另外维护 \lstinline{facpow} 数组. \lstinline{facpow[n]} 表示 $n$ 的最小的素因子组成的积, 这样就又将问题化为了两个子问题: 求函数关于某个素数幂次的值, 然后再合并该函数幂次和另一个和其互素的数的函数值.

对于数论函数使用 Euler 筛预处理预处理, 考虑三种情况:

\begin{enumerate}
  \item 参数为素数, 此时该数论函数应当有简单的函数表达.
  \item 参数为合数, 且当前筛到的素数不为其因子, 此时即为两个互素的数合并求函数值.
  \item 参数为合数, 且当前筛到的素数为其因子, 此时可以利用 \lstinline{facpow} 数组辅助求值.
\end{enumerate}

\lstinputlisting{NT/euler-sieve.hh}

\subsection{Euler 函数}
$\varphi(n)$ 是小于等于 $n$ 的整数中和 $n$ 互质的数个数. 可以用 Euler 线性筛预处理.
\[\varphi(n)=n\prod_{p\,\text{为}\,n\,\text{素因子}}\left(1-\frac1p\right)\]
\paragraph{性质:}
\begin{itemize}
  \item $\displaystyle\sum_{d\mid n}\varphi(d)=n$
  \item $a\mid b\implies\varphi(a)\mid\varphi(b)$
  \item $n\mid\varphi(a^n-1)\qquad\text{for }a,n>1$
  \item $\displaystyle\varphi(nm)=\varphi(n)\varphi(m)\frac{\varphi(\gcd(n,m))}{\gcd(n,m)}$
  \item $\varphi(n)\varphi(m)=\varphi(\gcd(n,m))\varphi(\operatorname{lcm}(n,m))$
  \item $\displaystyle\frac{\varphi(n)}n=\frac{\varphi(\operatorname{rad}n)}{\operatorname{rad}n}$, $\operatorname{rad}n$ 为 $n$ 所有不同的素因子的一次方的乘积.
  \item $\displaystyle\sum_{d\mid n}\frac{\mu^2(d)}{\varphi(d)}=\frac{\varphi(d)}d$
  \item $\displaystyle\sum_{\substack{1\le k\le n\\k\perp n}}k=\frac12[n=1]+\frac12n\varphi(n)$
  \item Fermat-Euler 定理 $a^{\varphi(n)}\equiv1\pmod n$
  \begin{itemize}
    \item $a^k\equiv a^{k\bmod\varphi(n)+\varphi(n)}\pmod n\qquad\text{for }a\not\perp n$
    \item $a^k\equiv a^{k\bmod\varphi(n)}\pmod n\qquad\qquad\;\!\text{for }a\perp n$
  \end{itemize}
\end{itemize}

\subsection{Möbius 函数}
\[
  \mu(n)=\begin{cases}
    -1&n\text{无平方数因数且素因子个数为奇数}\\
    0&n\text{有平方数因数}\\
    +1&n\text{无平方数因数且素因子个数为偶数}\\
  \end{cases}
\]
可用 Euler 线性筛预处理.

\paragraph{性质:}
\begin{itemize}
  \item $\displaystyle\sum_{d\mid n}\mu(d)=[n=1]$
  \item $\displaystyle\mu^2(n)=\sum_{d^2\mid n}\mu(d)$
\end{itemize}

\subsection{Möbius 反演}

对于数论函数 $\displaystyle f(n)=\sum_{d\mid n}g(d)$, 有 $\displaystyle g(n)=\sum_{d\mid n}\mu\left(\frac nd\right)f(d)$.

\paragraph{结论:}
\begin{itemize}
  \item $\displaystyle[i\perp j]=\sum_{d\mid\gcd(i,j)}\mu(d)$, 参见 Möbius 函数性质.
  \item $\displaystyle\varphi(n)=\sum_{d\mid n}d\mu\left(\frac nd\right)$, 见 Euler 函数性质.
  \item $\displaystyle\sum_{i=1}^n\sum_{j=1}^m[\gcd(i,j)=d]=\sum_{r=1}^{\left\lfloor\min\left\{\mathlarger{\frac nd,\frac md}\right\}\right\rfloor}\mu(r)\left\lfloor\frac n{rd}\right\rfloor\left\lfloor\frac m{rd}\right\rfloor$
  \item $\displaystyle\sum_{i=1}^n\sum_{j=1}^m\gcd(i,j)=\sum_{k=1}^{\min\{n,m\}}\varphi(k)\left\lfloor\frac nk\right\rfloor\left\lfloor\frac mk\right\rfloor$
\end{itemize}

\paragraph{例子:}
\[
  \begin{aligned}
    &\sum_{i=1}^n\sum_{j=1}^m\frac{ij}{\gcd(i,j)}\\
    =&\sum_{d=1}^{\min\{n,m\}}\frac1d\sum_{i=1}^ni\sum_{j=1}^mj[\gcd(i,j)=d]&&\triangleright\text{提取最大公约数}\ d\\
    =&\sum_{d=1}^{\min\{n,m\}}\frac1d\sum_{i=1}^{\left\lfloor\mathlarger{\frac nd}\right\rfloor}di\sum_{j=1}^{\left\lfloor\mathlarger{\frac md}\right\rfloor}dj[i\perp j]&&\triangleright\text{两个求和同除以}\ d\\
    =&\sum_{d=1}^{\min\{n,m\}}d\sum_{i=1}^{\left\lfloor\mathlarger{\frac nd}\right\rfloor}i\sum_{j=1}^{\left\lfloor\mathlarger{\frac md}\right\rfloor}j\sum_{r\mid\gcd(i,j)}\mu(r)&&\triangleright\text{参见上述结论}\\
    =&\sum_{d=1}^{\min\{n,m\}}\sum_{r=1}^{\left\lfloor\min\left\{\mathlarger{\frac nd,\frac md}\right\}\right\rfloor}d\mu(r)\sum_{i=1}^{\left\lfloor\mathlarger{\frac n{rd}}\right\rfloor}ri\sum_{j=1}^{\left\lfloor\mathlarger{\frac m{rd}}\right\rfloor}rj&&\triangleright\text{将}\ r\ \text{提到和}\ d\ \text{一起}\\
    =&\sum_{k=1}^{\min\{n,m\}}\sum_{r\mid k}kr\mu(r)\sum_{i=1}^{\left\lfloor\mathlarger{\frac nk}\right\rfloor}i\sum_{j=1}^{\left\lfloor\mathlarger{\frac mk}\right\rfloor}j&&\triangleright\text{枚举}\ k=rd\ \text{和其中一个因子}\\
  \end{aligned}
\]

上例最后, $\displaystyle\sum_{r\mid k}kr\mu(r)$ 为两个积性函数 ($\operatorname{id}^{-1}$ 和 $\mu$) 的 Dirichlet 卷积, 因此也是积性函数, 可以用 Euler 筛预处理. 若 $n,m$ 范围特别大, 可以用杜教筛优化复杂度. 具体参考 ``杜教筛''.

剩下的 $\displaystyle\sum_{i=1}^{\left\lfloor\mathlarger{\frac nk}\right\rfloor}i\sum_{j=1}^{\left\lfloor\mathlarger{\frac mk}\right\rfloor}j$ 为等差数列求和, 可以化为 $\displaystyle\frac14\left\lfloor\frac nk\right\rfloor\left\lfloor\frac mk\right\rfloor\left(\left\lfloor\frac nk\right\rfloor+1\right)\left(\left\lfloor\frac mk\right\rfloor+1\right)$.

该例具体实现参考下一页 ``杜教筛''.

\clearpage
\subsection{杜教筛}
对于积性函数 $f(n)$, 求 $\displaystyle F(n)=\sum_{k=1}^nf(k)$.

找到积性函数 $g(n)$ 使得 $g$ 的前缀和 $\displaystyle G(n)=\sum_{k=1}^ng(k)$ 以及 $f$ 和 $g$ 的 Dirichlet 卷积函数 $\displaystyle h=f*g$ 前缀和 $\displaystyle H(n)=\sum_{k=1}^nh(k)$ 容易计算, 有
\[F(n)=H(n)-\sum_{k=2}^nF\left(\left\lfloor\frac nk\right\rfloor\right)g(k)\]

递归分段求 $F$ 即可, 直接计算 $O(n^\frac34)$, 预处理前 $O(n^\frac23)$ 项时总体复杂度最低, 为 $O(n^\frac23)$.

\paragraph{结论:}
\begin{itemize}
  \item $\displaystyle\mathlarger\Phi(n)=\sum_{k=1}^n\varphi(k)=\frac12n(n+1)-\sum_{k=2}^n\mathlarger\Phi\left(\left\lfloor\frac nk\right\rfloor\right)$
  \item $\displaystyle M(n)=\sum_{k=1}^n\mu(k)=1-\sum_{k=2}^nM\left(\left\lfloor\frac nk\right\rfloor\right)$
\end{itemize}

\paragraph{例子:}

计算函数 $\displaystyle f(n)=\sum_{r\mid k}kr\mu(r)$ 前缀和 $\displaystyle F(n)=\sum_{k=1}^nf(k)$. 容易发现 $g(n)=n$ 和 $f$ 的 Dirichlet 卷积函数 $h(n)=1$, 所以有 $H(n)=n$, $G$ 和 $H$ 均可较为轻松地求得.

\begin{lstlisting}
long G(long n) {
  return n * (n + 1) / 2;
}

long F(long n) {
  static unordered_map<long, long> mem;
  if (n < maxn) return Fn[n];  // Fn 为预处理的前 O(n^2/3) 项 F.
  const auto m = mem.find(n);
  if (m != mem.end()) return m.second;
  long r = n;
  for (long i = 2, j; i <= n; i = j + 1) {
    j = n / (n / i);
    r -= F(n / i) * (G(j) - G(i - 1));
  }
  return mem[n] = r;
}

long solve(long n, long m) {
  long r = 0;
  for (long i = 1, j; i <= min(n, m); i = j + 1) {
    const long ni = n / i, mi = m / i, Sn = G(ni), Sm = G(mi);
    j = min(n / ni, m / mi);
    r += Sn * Sm * (F(j) - F(i - 1));
  }
  return r;
}
\end{lstlisting}

\subsection{洲阁筛 (Min\_25 筛)}

对于积性函数 $f(n)$, 求 $\displaystyle F(n)=\sum_{k=1}^nf(k)$. 洲阁筛要求对于素数 $p$ 和任意整数 $c$, $f(p^c)$ 为 $p$ 的多项式. 洲阁筛比杜教筛更普适, 时间复杂度 $O\left(\frac{n^{3/4}}{\log n}\right)$, 空间复杂度 $O(\sqrt n)$.

考虑函数 $\operatorname{m25}(n,k)$ 表示 Eratosthenes 筛已经筛去 $k$ 个素数之后剩下的数的 $f$ 函数的和:

\[\operatorname{m25}(n,k)=\operatorname{m25}(n,k-1)-f(p_k)[\operatorname{m25}(\lfloor n/p_k\rfloor,k-1)-\operatorname{m25}(p_k-1,k-1)]\]

注意到 $\lfloor n/p\rfloor$ 有 $O(\sqrt n)$ 种取值可以一并处理, $p-1$ 小于 $\sqrt n$ 可以预处理.

\subsection{Jacobi 四平方和定理}
$a^2+b^2+c^2+d^2=n$ 的自然数解的个数, 当 $n$ 为奇数时为 $8\sigma_0(n)$, 当 $n$ 为偶数时为 $24\sigma_0(n)$, 其中 $\sigma_0$ 为除数函数.
